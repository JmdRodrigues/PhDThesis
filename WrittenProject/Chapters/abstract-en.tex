%!TEX root = ../template.tex
%%%%%%%%%%%%%%%%%%%%%%%%%%%%%%%%%%%%%%%%%%%%%%%%%%%%%%%%%%%%%%%%%%%%
%% abstrac-en.tex
%% NOVA thesis document file
%%
%% Abstract in English([^%]*)
%%%%%%%%%%%%%%%%%%%%%%%%%%%%%%%%%%%%%%%%%%%%%%%%%%%%%%%%%%%%%%%%%%%%

\typeout{NT FILE abstrac-en.tex}%

As data analysts, we communicate with data in a meaningful way. We use methods and algorithms to let data speak for itself and "tell" us the information we need. It is not always easy to unveil meaningful information from time series, specially with the current advent of existing wearables and available biodevices it comes with even greater importance to have tools that help analysts explore, analyse and retrieve this information and reduce the burden of such a time-consuming process. 

In this thesis, it is explored how this communication with biosignals can be made to retrieve the information that we needed. Can we translate the \textit{language of biosignals}? Can we then use words to describe specific patterns and shapes in biosignals? Are there any meaningful structural changes in biosignals  that can be \textit{unveilled}? 

Novel methods in time series analysis are proposed for fast, useful and meaningful information retrieval, with more expressive and intuitive means of \textit{communication}. In this sense, we propose tools for (1) a practical and manageable way to automatically segment single-channel or multimodal biosignal data using a self-similarity matrix (SSM) computed with a signals' feature-based representation; (2) more expressive and intuitive pattern search strategies with (A) a novel symbolic representation of time series (SSTS), and (B) a word feature-vector used with operators (\textit{Quo}TS); and (3) classification of time series, based on the mentioned symbolic approach (HeaRTS). 

In topic (1) we were able to deliver lucid visual support in interpreting biosignals with SSM while performing accurate automatic segmentation with the help of novelty and similarity functions, ultimately moving towards automatic labelling. As in (2), the novel symbolic representation could successfully be used with a limited vocabulary for pattern search in several biosignals' use-cases.  These novel search mechanisms performed with good accuracy, but mostly increased intuition and expressiveness. Finally, we were able to perform classification tasks with a text-based representation of signal and standard text-mining classifiers with competitive F1-scores.


%Biosignal-based technology has been increasingly available in our daily life, being a critical information source that has been widely applied in, among others, biometrics, sports, health care, rehabilitation assistance, and edutainment.
%
%Continuous data collection from biodevices, also motivated in industrial environments with Industry 4.0, provides a valuable volume of information, which also brings challenges, since datasets of this 
%size can no longer be handled by trivial means. The expertise of data scientists in data mining and machine learning (ML) is now needed, being the usage of ML models and their quality heavily dependent on well curated and prepared data, which is a sensitive and time consuming process. 
%
%In this thesis, we explore novel methods in time series analysis for fast, useful and meaningful information retrieval, that are also more expressive and intuitive. In this sense, we propose tools for (1) a practical and manageable way to automatically segment and label single-channel or multimodal biosignal data using a self-similarity matrix (SSM) computed with a signals' feature-based representation; (2) more expressive and intuitive pattern search strategies with (A) a novel symbolic representation of time series (SSTS), and (B) a word feature-vector used with operators (\textit{Quo}TS); and (3) classification of time series, based on the mentioned symbolic approach (HeaRTS). 
%
%In (1) it was applied to public biosignal datasets and it delivered lucid visual support in interpreting the biosignals with the SSM while performing accurate automatic segmentation with the help of the novelty and similarity functions. The proposed method had an overall F1-score of 0.9 in a series of automatic biosignal segmentation tasks, significantly outperforming the state-of-the-art algorithms. As in (2), the novel methods were applied in several use-cases to demonstrate that with the limited vocabulary, it was possible to find desired patterns with good accuracy, but mostly good intuition and expressiveness. Finally, we were able to perform time series classification with a text-based representation of the signal and standard text-mining classifiers with competitive accuracies.

% Palavras-chave do resumo em Inglês
\begin{keywords}
biosignals, time series, similarity, symbolic, text, expressiveness, classification, pattern, features
\end{keywords} 
