%!TEX root = ../template.tex
%%%%%%%%%%%%%%%%%%%%%%%%%%%%%%%%%%%%%%%%%%%%%%%%%%%%%%%%%%%%%%%%%%%%
%% abstrac-en.tex
%% NOVA thesis document file
%%
%% Abstract in English([^%]*)
%%%%%%%%%%%%%%%%%%%%%%%%%%%%%%%%%%%%%%%%%%%%%%%%%%%%%%%%%%%%%%%%%%%%

\typeout{NT FILE abstrac-en.tex}%

As data scientists in the biomedical field, we strive to uncover the structure and underlying patterns within biosignals' data that allow gaining a clearer understanding of physiological phenomena. Methods and algorithms are used as means of interaction and \textit{communication} to make the data \textit{express} its content to retrieve the information we need. In the current data deluge with the advent of existing wearables and available biodevices, it is not always easy to unveil this meaningful information from biosignals. Hence, there is an emerging necessity to have tools that help analysts explore, analyse and retrieve this information and reduce the burden of such a time-consuming process. 

In this thesis, this communication of the analyst using biosignals is explored answering the following questions: Can we \textit{decipher} the \textit{language of biosignals}? Can we then use words to describe specific patterns and shapes in biosignals? Are there any meaningful time-dependent strcutural changes in biosignals that can be \textit{unveilled}? 

Novel methods in time series analysis are proposed for fast, useful and meaningful information retrieval, to pave the way for more expressive and intuitive means of \textit{communication}. More specifically, tools are proposed for (1) a practical way to automatically segment single-channel or multimodal biosignal data using a self-similarity matrix (SSM) computed with a signals' feature-based representation; (2) more expressive and intuitive pattern search strategies with (a) a novel symbolic representation of time series (SSTS), and (b) a word feature-vector used with operators (\textit{Quo}TS); and (3) classification of time series, based on the mentioned symbolic approach (HeaRTS). 

In topic (1) lucid visual support was delivered in interpreting biosignals with an SSM, while performing accurate automatic segmentation with the help of novelty and similarity functions, ultimately moving towards automatic labelling. As in (2.a), the novel symbolic representation was successful in deciphering the signal in several biosignals' use-cases, being used, such as (2.b) to perform pattern search with a simple vocabulary. These novel search mechanisms performed with good accuracy in detecting the desired patterns, but mostly with increased intuition and expressiveness. Finally, in (2.c) was showed that the signal could be translated into a text-based representation to perform classification task with standard text-mining classifiers.


%Biosignal-based technology has been increasingly available in our daily life, being a critical information source that has been widely applied in, among others, biometrics, sports, health care, rehabilitation assistance, and edutainment.
%
%Continuous data collection from biodevices, also motivated in industrial environments with Industry 4.0, provides a valuable volume of information, which also brings challenges, since datasets of this 
%size can no longer be handled by trivial means. The expertise of data scientists in data mining and machine learning (ML) is now needed, being the usage of ML models and their quality heavily dependent on well curated and prepared data, which is a sensitive and time consuming process. 
%
%In this thesis, we explore novel methods in time series analysis for fast, useful and meaningful information retrieval, that are also more expressive and intuitive. In this sense, we propose tools for (1) a practical and manageable way to automatically segment and label single-channel or multimodal biosignal data using a self-similarity matrix (SSM) computed with a signals' feature-based representation; (2) more expressive and intuitive pattern search strategies with (A) a novel symbolic representation of time series (SSTS), and (B) a word feature-vector used with operators (\textit{Quo}TS); and (3) classification of time series, based on the mentioned symbolic approach (HeaRTS). 
%
%In (1) it was applied to public biosignal datasets and it delivered lucid visual support in interpreting the biosignals with the SSM while performing accurate automatic segmentation with the help of the novelty and similarity functions. The proposed method had an overall F1-score of 0.9 in a series of automatic biosignal segmentation tasks, significantly outperforming the state-of-the-art algorithms. As in (2), the novel methods were applied in several use-cases to demonstrate that with the limited vocabulary, it was possible to find desired patterns with good accuracy, but mostly good intuition and expressiveness. Finally, we were able to perform time series classification with a text-based representation of the signal and standard text-mining classifiers with competitive accuracies.

% Palavras-chave do resumo em Inglês
\begin{keywords}
biosignals, time series, similarity, segmentation, meaning, structure, symbolic, text, expressiveness, classification, pattern, features
\end{keywords} 
