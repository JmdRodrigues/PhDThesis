%!TEX root = ../template.tex
%%%%%%%%%%%%%%%%%%%%%%%%%%%%%%%%%%%%%%%%%%%%%%%%%%%%%%%%%%%%%%%%%%%%
%% chapter10.tex
%% NOVA thesis document file
%%
%% Chapter with lots of dummy text
%%%%%%%%%%%%%%%%%%%%%%%%%%%%%%%%%%%%%%%%%%%%%%%%%%%%%%%%%%%%%%%%%%%%

\typeout{NT FILE chapter10.tex}%

\chapter{Conclusion}
\label{cha:Conclusion}



Future Work

- Associate the Bag of Words model weights with the shapes extracted. Save the subsequences and the words. Transform it in the bag of subsequences, and you can create a distance between the words. For instance, a Peak in Signal 1 might be different than a Peak in Signal 2. Or the signal might have the same dynamic (up down flat) but the up and down have a much different shape. By comparing the shapes, we might have an additional distance measure that makes the process more relevant...keep this in mind.

- Applying the cost matrix to the SSM to see what comes out of it. Would it help get the cyclic patterns in one time?

- Explore more NLP methods that could be used on the symbolic/textual representation of time series

- BERT for Time Series

- Profile with a summarization and overall report of time series that includes a more detailed description of the content of the data, more statistical information

- Test the systems developed with a control group to see if the tool is more or less expressive

