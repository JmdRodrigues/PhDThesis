%!TEX root = ../template.tex
%%%%%%%%%%%%%%%%%%%%%%%%%%%%%%%%%%%%%%%%%%%%%%%%%%%%%%%%%%%%%%%%%%%%
%% acknowledgements.tex
%% NOVA thesis document file
%%
%% Text with acknowledgements
%%%%%%%%%%%%%%%%%%%%%%%%%%%%%%%%%%%%%%%%%%%%%%%%%%%%%%%%%%%%%%%%%%%%

\typeout{NT FILE acknowledgements.tex}%

\begin{ntacknowledgements}

Este trabalho está longe de ser meu, é um esforço conjunto das pessoas e entidades que estiveram comigo até e aquando do traçar deste percurso, bem como de todas aquelas que se juntaram ao longo desta caminhada. Nada que esteja presente neste trabalho foi possível sem elas. Convosco partilhei dúvidas e frustrações, mas também grandes vitórias. Por isso, a todos vós, presto os meus mais profundos e sinceros agradecimentos.
Gostaria de começar por agradecer à Faculdade de Ciências e Tecnologias da Universidade NOVA de Lisboa por me ter dado o espaço e condições para a realização deste percurso, e às entidades financiadoras, Fundação para a Ciência e Tecnologia e Volkswagen Autoeuropa.
Um agradecimento especial à família do grupo de Biosignais. Nesse laboratório, passei por muitos momentos exigentes, mas sempre foi um porto seguro onde sabia que poderia contar com qualquer um deles. Começando por agradecer ao meu mentor, prof. Hugo Gamboa, o principal responsável por continuar o caminho da ciência. Apesar das minhas dúvidas, o prof. Hugo acabou por me convencer de que este seria o passo seguinte. Foi e será sempre para mim um guia, uma figura de referência, que valorizou, inspirou e apoiou, mas sobretudo, desafiou as ideias que se encontram neste trabalho. Dos muitos desafios, nem todos foram ultrapassados, ficando alguns no abstrato. Mas muitos maturaram e permitiram que um campo de curiosidade e procura fosse fértil, tornando o abstrato em concreto. Ajudou-me a construir o investigador e parte da pessoa que sou hoje. Estou-lhe eternamente grato por ter investido em mim e espero que tenha desfrutado tanto como eu deste percurso. O David, tornou-se um amigo e “irmão” mais velho, de suporte e companheirismo, imprescindível na minha carreira muito jovem. Obrigado por todos os conselhos, conversas, e momentos bem passados que tivemos. Quem diria que tudo começou com uma pintura de um armário…A Cátia, figura central do grupo, partilhamos o início do laboratório, com muito bricolage. Ensinou-me a descomplicar e a tomar melhores decisões, e também foi uma grande confidente. Ao Daniel, uma fonte de conhecimento, e sobretudo uma grande fonte de ceticismo, altamente necessário para nos pôr à prova e melhorarmos. Partilhei muitos bons momentos com ele na nossa estadia na Dinamarca. Ao Ricardo, um obrigado pela energia e motivação que trazia todos os dias ao laboratório. À Nafiseh, agradeço-lhe por fazer com que me conhecesse melhor. Ao Phillip, agradeço as boas conversas que tivemos e discussões maravilhosas sobre tantos temas. Ao Luís, figura motivadora e muito energética. Obrigado por aturar os meus desabafos e pelas idas a Porto Brandão. Aos alunos de mestrado que acompanhei ou colaborei de alguma forma, Guilherme, Diogo, António, às Saras, Lua e Rita, obrigado por me ensinarem tanto. Aos restantes recentes elementos, Mariana e Daniel, espero que desfrutem do vosso tempo no grupo e continuem a elevar o seu nome. Quero também agradecer aos nossos antigos vizinhos do lado, Paulo e Jorge, obrigado pelo vosso companheirismo, boa disposição e constante motivação. Foram peças muito importantes ao longo destes anos.
Ao longo deste caminho, acabei por ter a sorte de partilhar conhecimento com muita gente, e colaborei com muitas entidades. Um agradecimento vai para a Fraunhofer-AICOS, em especial ao Duarte, por todo o conhecimento partilhado, pela sua exigência e rigor, e por todas as nossas reuniões a discutir séries temporais. À PLUX, obrigado por providenciar equipamento sempre que necessário. Na Autoeuropa, um obrigado especial à equipa de gestão industrial e ergonomia, Pedro, Zé, Natércia, Jakie e em especial, ao Carlos, por ter sempre tanta disponibilidade e vontade em concretizar as suas ideias.  Um obrigado ao grupo de ciência computacional da University of California Riverside. Ryan, Rutuja, Jiasi, thank you so much for the great environment you provided, my stay would not have been so pleasant without you. A special thank you to Prof. Eamonn Keogh. You are a great scientific inspiration. I always followed your work with a great interest and understand that there is so much that I must learn. Thank you for receiving me so well, showing me around and spending time with me. My 



O futuro pode ser desconhecido, mas vislumbra-se pitoresco quando o imagino ao lado de todos vós. Um forte obrigado.



\end{ntacknowledgements}