%!TEX root = ../template.tex
%%%%%%%%%%%%%%%%%%%%%%%%%%%%%%%%%%%%%%%%%%%%%%%%%%%%%%%%%%%%%%%%%%%%
%% abstrac-pt.tex
%% NOVA thesis document file
%%
%% Abstract in Portuguese
%%%%%%%%%%%%%%%%%%%%%%%%%%%%%%%%%%%%%%%%%%%%%%%%%%%%%%%%%%%%%%%%%%%%

\typeout{NT FILE abstrac-pt.tex}%

Enquanto analistas de dados no ramo da engenharia biomédica, procuramos desvendar estruturas e padrões, inerentes aos biosinais, que permitem compreender fenómenos fisiológicos associados. Métodos e algoritmos são utilizados como meios de interação e comunicação de forma a permitir aos dados expressarem o seu conteúdo para extrair a informação de interesse. Nem sempre se torna simples retirar esta informação, sendo que, especialmente com o uso mais frequente dos biodispositivos \textit{vestíveis}, torna-se cada vez mais relevante ter acesso a ferramentas que ajudem os analistas a explorar, analisar, e recolher informação, para reduzir a dificuldade, tempo e custo deste inevitável processo.

Neste trabalho, a comunicação entre o analista e os biosinais é explorada, procurando responder às seguintes questões: Podemos \textit{decifrar} a \textit{linguagem dos biosinais}? Podemos usar palavras para descrever padrões e formas específicas dos biosinais? Haverão mudanças estruturais que podem ser desvendadas? 

Novos métodos são propostos para a análise mais rápida, útil e significativa das séries temporais, abrindo o caminho para um meio de comunicação mais expressivo e intuitivo entre o analista e os dados recolhidos. Mais especificamente, são propostas ferramentas para (1) a segmentação automática de biosinais uni ou multimodais, por meio de uma matriz de auto-similaridade (SSM), calculada a partir da representação das características do sinal; (2) procuras de padrões mais expressivas e intuitivas com (a) uma nova representação simbólica das séries temporais (SSTS), (b) palavras que representam características significativas usadas com operadores (\textit{QuoTS}); e (3) classificação de séries temporais com base na representação simbólica dos sinais (HearTS).

Em relação ao tópico (1) foi apresentado um suporte visual lúcido para interpretar os biosinais com a SSM, enquanto também é possível realizar a segmentação automático com o uso da função de novidade e similaridade, a caminho do processo de categorização automática. Em (2.a), a representação simbólica proposta mostra-se competente em diversos casos de uso de biosnais, tal como o método proposto em (2.b), no contexto de procura de padrões em biosinais, com a ajuda de um vocabulário simples. Estes dois métodos propostos apresentaram resultados sólidos na idenficação dos padrões desejados, sobretudo com mais intuição e expressivadade. Finalmente, em (2.c) foi apresentado que o sinal poderia ser transformado em texto e ser usado em problemas de classificação utilizando classificadores tradicionalmente usados em texto.

% Palavras-chave do resumo em Português
\begin{keywords}
biosinais, séries temporais, similaridade, segmentação, significado, estrutura, simbólico, texto, expressividade, classificação, padrão, características
\end{keywords}
% to add an extra black line
